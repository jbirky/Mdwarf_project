\documentclass[compress]{beamer}
\usetheme{sthlm}

%-=-=-=-=-=-=-=-=-=-=-=-=-=-=-=-=-=-=-=-=-=-=-=-=
%        LOADING BEAMER PACKAGES
%-=-=-=-=-=-=-=-=-=-=-=-=-=-=-=-=-=-=-=-=-=-=-=-=

\usepackage{
booktabs,
datetime,
dtk-logos,
graphicx,
multicol,
pgfplots,
ragged2e,
tabularx,
tikz,
wasysym,
animate
}

\pgfplotsset{compat=1.8}

\usepackage[utf8]{inputenc}
\usepackage[english]{babel}
\usepackage[T1]{fontenc}
\usepackage{newpxtext,newpxmath}
\usepackage{listings}

\lstset{ %
language=[LaTeX]TeX,
basicstyle=\normalsize\ttfamily,
keywordstyle=,
numbers=left,
numberstyle=\tiny\ttfamily,
stepnumber=1,
showspaces=false,
showstringspaces=false,
showtabs=false,
breaklines=true,
frame=tb,
framerule=0.5pt,
tabsize=4,
framexleftmargin=0.5em,
framexrightmargin=0.5em,
xleftmargin=0.5em,
xrightmargin=0.5em
}



%-=-=-=-=-=-=-=-=-=-=-=-=-=-=-=-=-=-=-=-=-=-=-=-=
%        LOADING TIKZ LIBRARIES
%-=-=-=-=-=-=-=-=-=-=-=-=-=-=-=-=-=-=-=-=-=-=-=-=

\usetikzlibrary{
backgrounds,
mindmap
}

%-=-=-=-=-=-=-=-=-=-=-=-=-=-=-=-=-=-=-=-=-=-=-=-=
%        BEAMER OPTIONS
%-=-=-=-=-=-=-=-=-=-=-=-=-=-=-=-=-=-=-=-=-=-=-=-=

\setbeameroption{show notes}

%-=-=-=-=-=-=-=-=-=-=-=-=-=-=-=-=-=-=-=-=-=-=-=-=
%        BEAMER COMMANDS
%-=-=-=-=-=-=-=-=-=-=-=-=-=-=-=-=-=-=-=-=-=-=-=-=


%-=-=-=-=-=-=-=-=-=-=-=-=-=-=-=-=-=-=-=-=-=-=-=-=
%
%	PRESENTATION INFORMATION
%
%-=-=-=-=-=-=-=-=-=-=-=-=-=-=-=-=-=-=-=-=-=-=-=-=

\title{Data-driven cannon Models for APOGEE M dwarfs}
\subtitle{\textcolor{sthlmDarkGreen}{Milky Way Group Meeting}}
\date{\small{16 August 2017}}
\author{\texttt{Jessica Birky - UC San Diego}}
\institute{\texttt{Advisor: David Hogg - NYU/MPIA/Simons}}

\hypersetup{
pdfauthor = {Egmon Pereira: @Egmon},      
pdfsubject = {Teologia, },
pdfkeywords = {},  
pdfmoddate= {D:\pdfdate},          
pdfcreator = {WriteLaTeX}
}

\begin{document}


\begin{frame}
	\titlepage
\end{frame}


\begin{frame}{Motivation}

\Large Data-driven models are physically informative
\begin{enumerate} \large
	\item[•] The Cannon allows us to systematically search for features/indices that are sensitive to physical parameters of interest
	\pause
	\item[•] Cannon successes:
	\begin{enumerate}
		\item[•] Stellar parameters $T_{eff}$, logg, [Fe/H] (Ness 2015)
		\begin{enumerate}
			\item[•] Currently implemented into the APOGEE DR14 pipeline
		\end{enumerate}
		\item[•] Survey 1 $\Rightarrow$ Survey 2 (LAMOST $\Rightarrow$ APOGEE, Ho 2015)
	\end{enumerate}
	\pause
	\item[•] Potential ideas for M dwarf regime:
	\begin{enumerate}
		\item[•] Optical $\Rightarrow$ Infrared (K band $\Rightarrow$ H band)
		\item[•] Magnetic activity (H$_{\alpha}$), any indices that would be interesting
	\end{enumerate}
\end{enumerate}

\end{frame}


\begin{frame}{Motivation}

\Large Potential with the \textcolor{sthlmDarkGreen}{APOGEE} survey
\begin{enumerate} \large
	\item[•] Specifications: H band (1.5 - 1.7 $\mu$m), R $\sim$ 22,000
	\item[•] aspcap pipeline does not attempt to model the M dwarf spectra
\end{enumerate}

\pause
\begin{enumerate} \large
	\item[$\Rightarrow$] There are many M dwarfs (1400+, Desphande 2013), most which are not labeled or classified
\end{enumerate}
\end{frame}


\begin{frame}{Background: The Cannon}
\Large 
Assumptions: 
\begin{enumerate} \large
	\item[•] Spectral with identical labels have identical looking spectra
	\item[•] Flux at each wavelength pixel varies smoothly with change in parameter
\end{enumerate}

\pause
Requirements:
\begin{enumerate} \large
	\item[•] \textcolor{sthlmDarkGreen}{Training Set:} set of objects with known parameters (ideally consistent and very accurate) 
	\item[•] \textcolor{sthlmDarkGreen}{Test Set:} all objects (with unknown or unreliable parameters) that lie within the parameter span of the training set
\end{enumerate}
\end{frame}


\begin{frame}{Background: The Cannon}
\Large 
Cannon Model:
\[\textcolor{sthlmDarkGreen}{f_{n\lambda} = \theta^{T} \cdot l_{n} + noise} \]

\pause
$\theta$: Spectral model coefficients (at each $\lambda$) \\
1D Cannon Model: 
\begin{enumerate} \large
	\item[•] $l_{n} = [1, SPT, SPT^{2}]$ 
\end{enumerate}
3D Cannon Model: 
\begin{enumerate} \large
	\item[•] $l_{n} = [1, Teff, log g, [Fe/H], Teff^{2}, Teff \cdot log g, Teff \cdot [Fe/H], log g^{2}, log g \cdot [Fe/H], [Fe/H]^{2}]$ 
\end{enumerate}
\end{frame}
 

\begin{frame}{Project Scope}
\large Constructing a training set
\begin{enumerate} \large
	\item[•] Literature sample: 44 sources from Simbad \textcolor{sthlmDarkGreen}{\{Teff, [Fe/H]\}}
	\begin{enumerate}
		\item[•] Inconsistent training labels, inconsistent results
	\end{enumerate}

	\pause
	\item[•] Rajpurohit 2017: 45 sources \textcolor{sthlmDarkGreen}{\{Teff, logg, [Fe/H]\}}, or \textcolor{sthlmDarkGreen}{\{Spectral type\}}
	\begin{enumerate}
		\item[•] Temperatures concentrated 3000-3300K range
		\item[•] Labels derived are inconsistent with training labels
	\end{enumerate}

	\pause
	\item[•] West 2011: 69/70,841 sources matched \textcolor{sthlmDarkGreen}{\{Spectral type\}}
	\begin{enumerate}
		\item[•] Most consistent training set
	\end{enumerate}

\end{enumerate}

\pause 
\textcolor{red}{Objective:} Spectral type all APOGEE M dwarfs using data-driven models  

\pause
\textcolor{sthlmDarkGreen}{Broad scope:} we hope in some way data-driven models can be used to improve physical ones.
\end{frame}


\begin{frame}{Selecting the Test Set}
Sorting by photometry
\begin{enumerate} \small
	\item[•] All APOGEE sources vs. 340 known M dwarfs (West 2011, Rajpurohit 2017, Desphande 2013, Aganze's brown dwarf xmatch)
\end{enumerate}
\begin{figure}
	\includegraphics[width=.7\textwidth, keepaspectratio]{photos/dwarfs_aspcap} 
\end{figure}
\end{frame}


\begin{frame}{Selecting the Test Set}
Comparison: sources labeled vs. not labeled by aspcap:
\begin{figure}
	\includegraphics[height=.45\textheight]{photos/aspcap_by_teff}  
	\includegraphics[height=.45\textheight]{photos/aspcap_by_logg}  
\end{figure}
\textcolor{sthlmDarkGreen}{Left}: all APOGEE sources colored by aspcap \textcolor{sthlmDarkGreen}{Teff} \\
\textcolor{sthlmDarkGreen}{Right}: all APOGEE sources colored by aspcap \textcolor{sthlmDarkGreen}{logg} 
\end{frame}


\begin{frame}{Selecting the Test Set}
Sorting by spectra
\begin{figure}
	\includegraphics[width=\textwidth, keepaspectratio]{photos/templates.pdf} 
	\caption{M dwarf vs. M giant}
\end{figure}
\end{frame}


\begin{frame}{Selecting the Test Set}
Template sorting and SNR cut
\begin{figure}
	\includegraphics[width=.45\textwidth]{photos/known_mdwarfs_sorted.png} 
	\includegraphics[width=.6\textwidth]{photos/Chi_Squared_Scatter_SNR.pdf} 
\end{figure}

\pause
Proceedure (in progress): 1) Color cut, 2) Sort by comparing to M dwarf and M giant templates, 3) Cut SNR > 100
\end{frame}


\begin{frame}{West Sample: Training with Spectral Types}
Sample Specifications
\begin{enumerate} \small
	\item[•] \textcolor{sthlmDarkGreen}{51 sources M0-M9} with SNR > 40
	\item[•] Consistent SPT labels obtained from SDSS spectra 
	\begin{enumerate} 
		\item[•] using The Hammer + visual inspection 
	\end{enumerate}
	\item[•] Spectral Types: not the most physically informative parameter, but useful start to indentifying M dwarfs and for later modelling Teff and [Fe/H]
\end{enumerate}


\begin{figure}
	\includegraphics[width=.5\textwidth]{photos/west_spt_dist} 
	\includegraphics[width=.5\textwidth]{photos/west_color_mag} 
\end{figure}
\end{frame}


\begin{frame}{West Spectral sequence}
\begin{figure}
	\includegraphics[width=.42\textwidth]{photos/desphande_spec_seq} 
	\includegraphics[width=.57\textwidth]{photos/west_spec_seq} 
\end{figure}
\small
\textcolor{sthlmDarkGreen}{Left}: Desphande 2013; Spectral Types apporoximated by binned V-J color
\hangindent=7.8mm Highlights: Spectral type sensitive regions identified by Desphande

\textcolor{sthlmDarkGreen}{Right}: 9 selected spectral with types from West 2011
\end{frame}


\begin{frame}{West Spectral sequence}
\begin{figure}
	\includegraphics[width=.5\textwidth]{photos/bochanski_spec_seq} 
	\includegraphics[width=.5\textwidth]{photos/west_sdss_spec_seq} 
\end{figure}
\end{frame}


\begin{frame}{West-trained sample: results}
\small \textcolor{sthlmDarkGreen}{SPT Scatter:} \textcolor{red}{0.617}
\begin{figure}
	\includegraphics[width=.55\textwidth]{photos/west_self_test} 
	\includegraphics[width=.45\textwidth]{photos/west_crv} 
\end{figure}
\small \textcolor{sthlmDarkGreen}{Left}: Self-consistency test 
\footnotesize
\hangindent=15mm -- train on 51 spectra, re-derive labels for all 51 spectra \\
\small
\textcolor{sthlmDarkGreen}{Right}: Cross-validation test
\footnotesize
\hangindent=15mm -- leave out spectrum n, train on 50 spectra, derive label for spectrum n; repeat for all 51 spectra \\
\end{frame}


\begin{frame}{West-trained sample: results}
Training with magnetic activity (including EW of H$_{\alpha}$) \\
\small \textcolor{sthlmDarkGreen}{SPT Scatter:} \textcolor{red}{0.556}
\begin{figure}
	\includegraphics[width=.5\textwidth]{photos/west_2_1} 
	\includegraphics[width=.5\textwidth]{photos/west_2_2} 
\end{figure}
\small \textcolor{sthlmDarkGreen}{Left}: Teff Self-consistency test; \textcolor{sthlmDarkGreen}{Right}: H$_{\alpha}$ Self-consistency test
\end{frame}


\begin{frame}{West-trained models: Spectral sequence}
\begin{figure}
	\includegraphics[width=.5\textwidth]{photos/desphande_spec_seq} 
	\includegraphics[width=.52\textwidth]{photos/cannon_spec_seq} 
\end{figure}
\small \textcolor{sthlmDarkGreen}{Left}: Desphande 2013; \textcolor{sthlmDarkGreen}{Right}: West-trained Cannon models
\end{frame}


\begin{frame}{Spectral sequence: data vs. Cannon models}
\begin{figure}
	\includegraphics[width=\textwidth]{photos/1_Spectral_Sequence.pdf} 
\end{figure}
\end{frame}


\begin{frame}{Spectral sequence: data vs. Cannon models}
\begin{figure}
	\includegraphics[height=.45\textheight]{photos/2_Spectral_Sequence.pdf}  \\
	\includegraphics[height=.45\textheight]{photos/3_Spectral_Sequence.pdf}  
\end{figure}
\end{frame}


\begin{frame}{Rajpurohit-trained Cannon Models}
Sample specifications
\begin{enumerate} 
	\item[•] \textcolor{sthlmDarkGreen}{45 sources} with labels \textcolor{sthlmDarkGreen}{\{Teff, logg, [Fe/H], Spectral type\}}
	\item[•] Labels derived from BT-Settl model grid (Homeier, unpublished)
\end{enumerate}

2 label results: \textcolor{sthlmDarkGreen}{\{Teff, [Fe/H], Spectral type\}}
\begin{figure}
	\includegraphics[height=.4\textheight]{photos/raj_2_1}  
	\includegraphics[height=.4\textheight]{photos/raj_2_2}   
\end{figure}
\end{frame}


\begin{frame}{Rajpurohit-trained Cannon Models}
3 label results: \textcolor{sthlmDarkGreen}{\{Teff, logg, [Fe/H], Spectral type\}}
\begin{figure}
	\includegraphics[height=.4\textheight]{photos/raj_3_1}  \\
	\includegraphics[height=.4\textheight]{photos/raj_3_2}  
	\includegraphics[height=.4\textheight]{photos/raj_3_3} 
\end{figure}
\end{frame}


\begin{frame}{Summary}
We have consistent results for spectral-typing APOGEE M dwarfs. \\

Current {Teff, [Fe/H]} training set does not span parameter range well enough
\begin{enumerate} 
	\item[•] But features are modelled well and may be useful for precise RVs/finding binaries
\end{enumerate}

Fixing systematic issues within The Cannon/aspcap may improve model fits
\begin{enumerate} 
	\item[•] Cannon optimizer is often converging to a local minimum
	\item[•] Continuum normalization producing artificial spikes in the data
	\item[•] Removing outliers in polynomial interpolation
\end{enumerate}
\end{frame}


\section{Model Fits}


\begin{frame}{Model fits: 2M11091225-0436249}
Rajpurohit-trained model synthesized to \textcolor{sthlmDarkGreen}{Training Labels}
\begin{figure}
	\includegraphics[width=\textwidth]{photos/train_fit}   
\end{figure}
Rajpurohit-trained model synthesized to \textcolor{sthlmDarkGreen}{Test Labels}
\begin{figure}
	\includegraphics[width=\textwidth]{photos/test_fit}   
\end{figure}
\small
Notice $\chi^{2}$ of test fit is lower than $\chi^{2}$ of training fit -- there is a bug in The Cannon optimizer, but at least the model coefficients are not skewed at the edge of the training set.
\end{frame}


\section{Future Work}


\begin{frame}{Future Work} \large
Would ideally like to train on 3 stellar parameters (Teff, logg, [Fe/H]) -- constructing a complete training data set is the biggest challenge
\begin{enumerate} \normalsize
	\item[•] Ideally my training set would be larger in number and span a larger parameter range
	\item[•] APOGEE was not designed to study cool main-sequence stars (rather giant and hotter stars) - APOGEE lacks benchmark sources for M dwarfs
\end{enumerate}

\pause
Solutions:
\begin{enumerate} \normalsize
	\item[•] Model more M dwarfs with unknown parameters
	\item[•] Obtain more known sources (plans for SpeX/NIRSPEC observing with Adam Burgasser)
\end{enumerate}
\end{frame}


\begin{frame}{Future Work/Questions} \large
Obtaining accurate metallicities: \textcolor{sthlmDarkGreen}{FGK-M dwarf pairs}
\begin{enumerate} \normalsize
	\item[•] Use FGK pairs as the training set (but only 5 overlap with APOGEE -- xmatch Newton + Neves samples with DR14)
	\pause
	\item[•] Use empirical calibrations derived in these studies
\end{enumerate}
\small
Terrien 2012: [Fe/H] = .34(EW$_{Ca}$) + .407(EW$_{K}$) + .436(H$_{2}$O - H) - 1.485
\begin{figure}
	\includegraphics[width=\textwidth]{photos/bonfils_table}
	\caption{M dwarf comparative study of photometric metallicity scales (Neves 2011)}   
\end{figure}
\end{frame}


\begin{frame}{Training parameters} 
The Cannon (in principle) can be trained on any label/parameter.
Parameters we have:
	\begin{center}
	\begin{tabular}{ |c|c|c| } 
	 \hline
	 Parameter & West Sample & Rajpurohit Sample \\ 
	 \hline
	 \hline
	 Spectra & APOGEE 	& APOGEE \\ 
	 		 & SDSS 	&  \\ 
	 \hline 
	 Stellar & Spectral Type & Spectral Type \\
	 Parameters & & Teff, logg, [Fe/H] \\
	 \hline
	 Photometry & JHK, ugriz & JHK \\
	 \hline
	 Distance   & Photometric estimates & -- \\
	 \hline
	 EW & H$_{\alpha}$, H$_{\beta}$, H$_{\gamma}$ & -- \\
	 measurements & TiO$_{5}$, TiO$_{2}$  & -- \\
	 \hline
	\end{tabular}
	\end{center}
\end{frame}


\section{End}


\end{document}


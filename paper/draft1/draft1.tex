\documentclass[modern]{aastex62}

\usepackage{color}
\usepackage{amsmath}
\usepackage{enumitem}
\usepackage{graphicx}


\definecolor{bcolor}{RGB}{0, 51, 153}
\hypersetup{linkcolor=bcolor,citecolor=bcolor,filecolor=cyan,urlcolor=bcolor}

\newcommand{\vdag}{(v)^\dagger}
\newcommand\aastex{AAS\TeX}
\newcommand\latex{La\TeX}

\received{\color{red} MM/DD/YY}
\revised{\color{red} MM/DD/YY}
\accepted{\color{red} MM/DD/YY}
\submitjournal{\color{red} JOURNAL}

\shorttitle{Birky \& Hogg}
\shortauthors{Birky \& Hogg}


\begin{document}

\title{DATA DRIVEN MODELS FOR APOGEE M DWARFS}


\correspondingauthor{Jessica Birky}
\email{jbirky@ucsd.edu}

\author[0000-0002-7961-6881]{Jessica Birky}
\affil{Center for Astrophysics and Space Science, University of California San Diego, La Jolla, CA 92093, USA}
\affil{Max-Planck-Institut f\"ur Astronomie, K\"onigstuhl 17, D-69117 Heidelberg}

\author[0000-0003-2866-9403]{David W. Hogg}
\affil{Max-Planck-Institut f\"ur Astronomie, K\"onigstuhl 17, D-69117 Heidelberg}
\affiliation{Center for Computational Astrophysics, Flatiron Institute, 162 Fifth Ave, New York, NY 10010, USA}
\affiliation{Center for Cosmology and Particle Physics, Department of Physics, New York University, 726
Broadway, New York, NY 10003, USA}
\affil{Center for Data Science, New York University, 60 Fifth Ave, New York, NY 10011, USA}


\begin{abstract}

The Cannon \citep{Ness:2015} is a flexible, data-driven spectral modeling and parameter inference framework, demonstrated on high-resolution Apache Point Galactic Evolution Experiment (APOGEE; $\lambda/\Delta\lambda\sim22,500$, $1.5-1.7 \mu m$) spectra of giant stars to estimate stellar labels (T$_{\rm eff}$, logg, [Fe/H], and chemical abundances) to precisions higher than the model-grid pipeline. The lack of reliable stellar parameters reported by the APOGEE pipeline for temperatures less than $\sim3550K$ \citep{Schmidt:2016}, motivates the extension of this approach to M dwarf stars. Using a training set of 51 M dwarfs with spectral types ranging M0-M9 obtained from SDSS optical spectra, we demonstrate that The Cannon can infer spectral types to a precision of 0.6 types. We then use 30 M dwarfs ranging $3072 < T_{\rm eff} < 4131K$, and -0.48 $<$ [Fe/H] $<$ 0.49 to train a two-parameter model precise to 44K and 0.05 dex respectively. Additionally we discuss the extension of a model to other labels, and the scientific objectives a data-driven pipeline could enable.

\end{abstract}

\keywords{stars: Mdwarfs -- stellar parameters}
 
%%==========================================================================================
\section{Introduction} \label{sec:intro}

% \begin{enumerate}
% \item[-] Why M-dwarfs are important: most abundant type of star, are planet hosts (The CARMENES search for exoplanets around M dwarfs \citealt{Trifonov:2018}; some famous examples: Trappist 1 (M8) -- \citealt{Gillon:2017}; Kepler-186, Kepler-138... other examples?), have lifespans longer than the age of the universe (billions or up to trillions of years) \citep{Laughlin:1997} making them an interesting probe of Milky way populations \citep{Bochanski:2007}.
% \end{enumerate}
Very low mass (VLM) are the most abundant type of star, comprising $\sim 70 \%$ of the galaxy's population by number \citep{Bochanski:2010} and $\sim 40 \%$ by mass are known to have lifespans longer than the age of the universe (billions or up to trillions of years) \citep{Laughlin:1997} making them an interesting probe of Milky way populations \citep{Bochanski:2007}.

\subsection{M dwarf Challenge}
\begin{enumerate}
\item[-] Various attempts / samples for which people have tried to make substantial catalogs of M-dwarfs with physical labels on them (\citealt{West:2011} is the largest M dwarf catalog of $\sim$ 70k sources; \citealt{Deacon:2014} -- \color{red}{Need citations!}\color{black}) 

\item[-] Why is there a problem labeling M dwarfs? Generally, two kinds of approaches to labeling:

 (1) model the spectra using atmospheric models (BTSettl, \citealt{Allard:2011}; PHOENIX, \citealt{Husser:2013}, MARCS \citealt{Gustafsson:2008}, ATLAS \citealt{Castelli:2004}, etc.) find some places in literature where people have tried these on M dwarfs (\color{red}{Need citations... }\color{black} look at Burgasser, Blake...) - however missing opacities, and complexities due to molecules/clouds/chemistry make this difficult (\color{red}{Need citations... }\color{black} \citealt{Allard:2013}); 

 (2) model narrow wavelegth region where models are good, look at specific features (\citealt{Rojas-Ayala:2012}...), or reduce resolution (\citealt{Casagrande:2008} estimated effective temperatures, bolometric luminosities, and metallicities using optical and infrared photometries) -- however we want to use \emph{all} of the data!

% \item[-] The idea in general of The Cannon, and the fact that it could be used to amplify a small number of labeled M-dwarfs in a training set into a huge number of labeled M-dwarfs, provided that there are overlapping objects. Make the point that The Cannon does not require us to have good M-dwarf models in the APOGEE wavelength range at all!

\item[-] Why The Cannon approach? Can be used to amplify a small number of labeled M-dwarfs in a training set into a huge number of labeled M-dwarfs, provided that there are overlapping objects; Make the point that The Cannon does not require us to have good M-dwarf models in the APOGEE wavelength range at all!

\item[-] Describe M dwarf studies that have already been conducted in APOGEE (\citealt{Schmidt:2016} -- analyzed trends and reliablility of ASPCAP parameters for K/M dwarfs; \citealt{Souto:2017} -- modeled 2 M dwarfs, determined Teff/logg/metallicity + 13 elemental abundances; \citealt{Desphande:2013} -- modeled rotational and radial velocites for several hundred M dwarfs; \citealt{Rajpurohit:2018} -- tested BTSettl model grids on 45 M dwarfs, estimated Teff/logg/metallicity)

\end{enumerate}

A number of M dwarf studies


%%==========================================================================================
\section{Data} \label{sec:data}

\subsection{APOGEE Survey}
% \begin{enumerate}
% \item[-] Describe the APOGEE data (survey mission: \citealt{Majewski:2015}; target selection: \citealt{Zasowski:2013}; ASPCAP pipeline: \citealt{Perez:2016}). 

% \item[-] Some M dwarfs in APOGEE proposed as ancilliary targets (Blake/Covey sources), plus an unknown number observed unintentionally \citep{Desphande:2013}.
% \end{enumerate}

The APOGEE survey \citep{Majewski:2015} of the Sloan Digital Sky Survey IV (SDSS-IV) is a high resolution (R$\sim$22,500), H-band (1.5-1.7$\mu$m) survey which has observed over 260,000 stellar spectra since its fourteenth data release (DR14; \citealt{Abolfathi:2017}). Fundamental parameters for each of these stars are estimated by the APOGEE Stellar Parameter and Chemical Abundances Pipeline (ASPCAP; \citealt{Perez:2016}), which employs a $\chi^2$ fitting procedure using the FERRE code to fit ATLAS9 radiative transfer models \citep{Castelli:2004} to determine atmospheric parameters, 15 chemical abundances and micro-turbulence parameters (T$_{\rm eff}$, log g, [Fe/H], [α/M], [C/M], [N/M]; \citealt{Meszaros:2012}). 

APOGEE is primarily targeted for bright stellar populations, with de-reddened photometry and color cutoffs of $7 \leq H \leq 13.8$ and $[J-K]_0 \geq 0.5$ \citep{Zasowski:2013} with the objective of studying galactic composition and evolution. However numerous cool, main sequence sources have also been observed either accidentally, or as targets proposed by the M dwarf ancilliary survey \citep{Desphande:2013} (specific estimate based on Gaia?).


\subsection{Training Samples}

The Cannon is a \emph{generative model} which parameterizes the flux at each pixel of a spectrum in terms of a set of stellar labels (a flexible number of parameters chosen by the user; described in more detail in Section \ref{sec:cannon}). The model in this sense is used to \emph{transfer} labels from spectra which we know parameters for to those which we do not. Hence this data-driven approach effectively removes the challenges of physically modeling the atmosphere of a star (and common issues associated such as incomplete line lists or opacities), provided that we have a subset of spectra in the dataset with known (ideally very accurately measured) \emph{reference labels} that can be measured from other surveys/methods that are easier to infer from. 

For the purpose of this study we have constructed two different training samples: first a one-dimensional \emph{spectral type model}, and second a two-dimensional \emph{physical parameter model} which describes the temperature and metallicity. The choice of training labels, dimensionality of our data set, and requirements for a good training set we discuss further in Section \ref{sec:discussion}.

The spectral type sample consists of 51 sources, spanning M0-M9 (Figure \ref{fig:train_dist}) cross-matched from the \citealt{West:2011} catalog of 78,841 M dwarfs from SDSS. For each source in the catalog, spectral types were determined both through an automated routine using The Hammer \citep{Covey:2007} and by visual inspection.

The physical parameter sample consists of 30 sources distributed over $3072 < T_{\rm eff} < 4131K$, and -0.48 $<$ [Fe/H] $<$ 0.49 (Figure \ref{fig:train_dist}) drawn from \citealt{Mann:2015}.
\color{red}{why logg not included as atmospheric parameter?}\color{black}

\begin{figure}[ht]
\plottwo{figures/west_train_dist.png}{figures/mann_train_dist.png}
\caption{Spectral type model: training label vs. test label.} \label{fig:west_selftest} \label{fig:train_dist}
\end{figure}

\subsection{Data Processing}
Describe ASPCAP spectra files and Ness continuum normalization procedure.


%%==========================================================================================
\section{Methods} \label{sec:cannon}

\subsection{The Cannon: A Data-Driven Approach}

\begin{enumerate}
\item[-] A few-paragraph summary of what The Cannon does, with references (\citealt{Ness:2015}, \citealt{Casey:2016}, \citealt{Ho:2017a}, \citealt{Ho:2017b})
\end{enumerate}

Constructing such a model requires two steps: first the \emph{training step} in which the generative model is constructed at each pixel from the set of reference labels and fluxes, and second the \emph{test step} in which the model is applied to determine the labels of a spectrum 

% The key advantages of The Cannon being:
% Data-driven models take away the challenge of directly infering labels from a survey, instead we \emph{transfer labels} from another survery that is more accurate or easier to model.

\subsection{Data-Driven Model}
\begin{enumerate}
\item[-] Describe assumptions of the model: (1) Sources with identical labels have near-identical flux at each pixel; (2) Expected flux at each pixel varies continuosly with change in label. 
\item[-] Briefly describe modeling procedure (training step, test step, quadratic labels, etc.), refer to \citealt{Ness:2015} for full description.
\item[-] Briefly describe model inputs and outputs (1D spectral type model and 2D teff/metallicity and label vectors for those). Mention other models we've tested.
\end{enumerate}


%%==========================================================================================
\section{Experimental Results} \label{sec:results}

\subsection{Spectral Type Model}

\begin{enumerate}
\item[-] We trained The Cannon on 51 M dwarfs with a one-dimensional spectral type label, and was able to achieve a precision of $\pm$0.8 spectral types, more precise than the original training label uncertainty of $\pm$1 spectral type.
\item[-] Two tests of consistency were used to assess the validity of our derived test labels: Figure \ref{fig:west_selftest} displays the label self-test for the spectral type model, and Figure 2 displays 
\end{enumerate}


\begin{figure}[ht]
\plotone{figures/west_spt_self_test.png}
\caption{Spectral type model: training label vs. test label.} \label{fig:west_selftest}
\end{figure}


\begin{figure}[ht]
\plotone{figures/1_Spectral_Sequence.pdf}
\plotone{figures/2_Spectral_Sequence.pdf}
\plotone{figures/3_Spectral_Sequence.pdf}
\caption{ Spectral sequence of dwarfs in training set M0-M9; chip 1 of APOGEE
spectrum with highlighted spectral type sensitive regions identified in \citealt{Desphande:2013}.} \label{fig:sp_sequence}
\end{figure}


\begin{enumerate}
\item[-] For the physical parameter model we trained The Cannon on 30 M dwarfs with a two-dimensional with temperature/metallicity labels, and was able to achieve precisions of 44K/0.05dex respectively, more precise than both of the original training label uncertainties of 60K/0.08dex.

\item[-] Two tests of consistency were used to assess the validity of our derived test labels: Figure \ref{fig:west_selftest} displays the label self-test for the spectral type model, and Figure 2 displays 
\end{enumerate}



\subsection{Temperature/Metallicity Model}
\begin{figure}[ht]
\plottwo{figures/mann_teff_self_test.png}{figures/mann_feh_self_test.png}
\caption{Temperature and Metallicity label self test.} \label{fig:mann_selftest}
\end{figure}

\begin{figure}[ht]
\plotone{figures/1_physical_model.png}
\plotone{figures/2_physical_model.png}
\plotone{figures/3_physical_model.png}
\caption{\textit{Top two panels:} Mann-trained model for varying temperatures; \textit{bottom two panels:} Mann-trained model for varying metallicities.} \label{fig:mann_demo}
\end{figure}


%%==========================================================================================
\section{Discussion} \label{sec:discussion}

\begin{enumerate}
\item[-] Brief summary of what we did and found

\item[-] Things we learned along the way: For example, not all training sets worked well in our experiments, probably because they were too small or had too little dynamic range.

\item[-] Some discussion of precision and accuracy: The Cannon can produce precise results, but is only as accurate as its training set. 

\item[-] Some discussion of whether the derivatives are sensible or surprising. \color{red}{Add derivative plots.}\color{black}

\item[-] Comment on specific features? \color{red}{Can easily plot some line lists, but not sure where to go with the analysis... What features are important for M dwarfs? Compare derivative plots of specific lines between theory-trained models and data-trained models?}\color{black}

\item[-] The Cannon as a test of evaluating training parameter accuracy? 
\end{enumerate}

\begin{figure}[ht]
\plotone{figures/west_spt_derivative.png}
\caption{Derivative plots for spectral type model.} \label{fig:west_derivative}
\end{figure}

\begin{figure}[ht]
\plotone{figures/mann_teff_derivative.png}
\plotone{figures/mann_feh_derivative.png}
\caption{Derivative plots for physical parameter model. \textit{Top:} Derivative of flux with repsect to temperature. \textit{Bottom:} Derivative of flux with repsect to metallicity.} \label{fig:mann_derivative}
\end{figure}


%%==========================================================================================
\clearpage
\bibliographystyle{aasjournal}
\bibliography{ref}

\end{document}



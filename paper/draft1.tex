\documentclass[modern]{aastex62}

\usepackage{color}
\usepackage{amsmath}
\usepackage{enumitem}


\definecolor{bcolor}{RGB}{0, 51, 153}
\hypersetup{linkcolor=bcolor,citecolor=bcolor,filecolor=cyan,urlcolor=bcolor}

\newcommand{\vdag}{(v)^\dagger}
\newcommand\aastex{AAS\TeX}
\newcommand\latex{La\TeX}

\received{\color{red} MM/DD/YY}
\revised{\color{red} MM/DD/YY}
\accepted{\color{red} MM/DD/YY}
\submitjournal{\color{red} JOURNAL}

\shorttitle{Birky \& Hogg}
\shortauthors{Birky \& Hogg}


\begin{document}

\title{DATA DRIVEN MODELS FOR APOGEE M DWARFS}


\correspondingauthor{Jessica Birky}
\email{jbirky@ucsd.edu}

\author[0000-0002-7961-6881]{Jessica Birky}
\affil{Center for Astrophysics and Space Science, University of California San Diego, La Jolla, CA 92093, USA}
\affil{Max-Planck-Institut f\"ur Astronomie, K\"onigstuhl 17, D-69117 Heidelberg}

\author[0000-0003-2866-9403]{David W. Hogg}
\affil{Max-Planck-Institut f\"ur Astronomie, K\"onigstuhl 17, D-69117 Heidelberg}
\affiliation{Center for Computational Astrophysics, Flatiron Institute, 162 Fifth Ave, New York, NY 10010, USA}
\affiliation{Center for Cosmology and Particle Physics, Department of Physics, New York University, 726
Broadway, New York, NY 10003, USA}
\affil{Center for Data Science, New York University, 60 Fifth Ave, New York, NY 10011, USA}


\begin{abstract}

The Cannon \citep{Ness:2015} is a flexible, data-driven spectral modeling and parameter inference framework, demonstrated on high-resolution Apache Point Galactic Evolution Experiment (APOGEE; $\lambda/\Delta\lambda\sim22,500$, $1.5-1.7 \mu m$) spectra of giant stars to estimate stellar labels (T$_{\rm eff}$, logg, [Fe/H], and chemical abundances) to precisions higher than the model-grid pipeline. The lack of reliable stellar parameters reported by the APOGEE pipeline for temperatures less than $\sim3550K$ \citep{Schmidt:2016}, motivates the extension of this approach to M dwarf stars. Using a training set of 51 M dwarfs with spectral types ranging M0-M9 obtained from SDSS optical spectra, we demonstrate that The Cannon can infer spectral types to a precision of 0.6 types. We then use 30 M dwarfs ranging $3072 < T_{\rm eff} < 4131K$, and -0.48 $<$ [Fe/H] $<$ 0.49 to train a two-parameter model precise to 44K and 0.05 dex respectively. Additionally we discuss the extension of a model to other labels, and the scientific objectives a data-driven pipeline could enable.

\end{abstract}

\keywords{stars: Mdwarfs -- stellar parameters}
 
%%==========================================================================================
\section{Introduction} \label{sec:intro}

\begin{enumerate}
\item[-] Why M-dwarfs are important: most abundant type of star, are planet hosts (Trappist 1, other examples?), have lifespans longer than the age of the universe (billions or up to trillions of years) making them an interesting probe of Milky way populations. \color{red}{Need citations!}
\end{enumerate}

\subsection{M dwarf Challenge}
\begin{enumerate}
\item[-] Various attempts / samples for which people have tried to make substantial catalogs of M-dwarfs with physical labels on them (\citealt{West:2011} is the largest M dwarf catalog of $\sim$ 70k sources... other catalogs? -- \color{red}{Need citations!}\color{black}) 

\item[-] Why is there a problem labeling M dwarfs? Generally, two kinds of approaches to labeling: (1) model the spectra using atmospheric models (BTSettl, PHOENIX, KURUCZ, MARCS, ATLAS, etc.) find some places in literature where people have tried these on M dwarfs (\color{red}{Need citations... }\color{black} look at Burgasser, Blake...) - however missing opacities, and complexities due to molecules/clouds/chemistry make this difficult (\color{red}{Need citations... }\color{black} look at Allard, Homeier...); (2) model narrow wavelegth region where models are good, look at specific features (\citealt{Rojas-Ayala:2012}...), or reduce resolution (use photometries) -- however we want to use \emph{all} of the data!

% \item[-] The idea in general of The Cannon, and the fact that it could be used to amplify a small number of labeled M-dwarfs in a training set into a huge number of labeled M-dwarfs, provided that there are overlapping objects. Make the point that The Cannon does not require us to have good M-dwarf models in the APOGEE wavelength range at all!

\item[-] Why The Cannon approach? Can be used to amplify a small number of labeled M-dwarfs in a training set into a huge number of labeled M-dwarfs, provided that there are overlapping objects; Make the point that The Cannon does not require us to have good M-dwarf models in the APOGEE wavelength range at all!

\end{enumerate}


%%==========================================================================================
\section{Data} \label{sec:data}

\subsection{APOGEE Survey}
\begin{enumerate}
\item[-] Describe the APOGEE data (survey mission: \citealt{Majewski:2015}; target selection: \citealt{Zasowski:2013}; ASPCAP pipeline: \citealt{Perez:2016}). 

\item[-] Some M dwarfs in APOGEE proposed as ancilliary targets (Blake/Covey sources), plus an unknown number observed unintentionally \citep{Desphande:2013}.

\item[-] Describe M dwarf studies that have already been conducted in APOGEE (\citealt{Schmidt:2016} -- analyzed trends and reliablility of ASPCAP parameters for K/M dwarfs; \citealt{Souto:2017} -- modeled 2 M dwarfs, determined Teff/logg/metallicity + 13 elemental abundances; \citealt{Desphande:2013} -- modeled rotational and radial velocites for several hundred M dwarfs; \citealt{Rajpurohit:2018} -- tested BTSettl model grids on 45 M dwarfs, estimated Teff/logg/metallicity)
\end{enumerate}

\subsection{Training Samples}
\begin{enumerate}
\item[-] Describe the spectral-type training data \citep{West:2011}. One paragraph plus citations. Include any cuts we apply. \color{red}{Make diagnostic plots.}\color{black}

\item[-] Describe the Teff, Fe/H training data \citep{Mann:2015}. Same; Include any cuts. \color{red}{Make diagnostic plots.}\color{black}
\end{enumerate}


%%==========================================================================================
\section{Methods} \label{sec:cannon}

\subsection{The Cannon: A Data-Driven Approach}
\begin{enumerate}
\item[-] A few-paragraph summary of what The Cannon does, with references (\citealt{Ness:2015}, \citealt{Casey:2016}, \citealt{Ho:2017a}, \citealt{Ho:2017b})
\end{enumerate}

\subsection{Data-Driven Model}
\begin{enumerate}
\item[-] Describe assumptions of the model: (1) Sources with identical labels have near-identical flux at each pixel; (2) Expected flux at each pixel varies continuosly with change in label. 
\item[-] Briefly describe modeling procedure (training step, test step, quadratic labels, etc.), refer to \citealt{Ness:2015} for full description.
\item[-] Briefly describe model inputs and outputs (1D spectral type model and 2D teff/metallicity and label vectors for those). Mention other models we've tested.
\end{enumerate}


%%==========================================================================================
\section{Experimental Results} \label{sec:results}
% \begin{enumerate}
% \item[-] The spectral-type Cannon. Show the results on your AAS poster, and also show plots of derivatives wrt type, with maybe blow-ups on interesting features.

% \item[-] The parameter Cannon. Same.
% \end{enumerate}

\subsection{Spectral Type Model}
\begin{figure}[ht]
\plotone{figures/west_spt_self_test.png}
\caption{Spectral type label self test.}
\end{figure}

\begin{figure}[ht]
\plotone{figures/1_Spectral_Sequence.pdf}
\plotone{figures/2_Spectral_Sequence.pdf}
\plotone{figures/3_Spectral_Sequence.pdf}
\caption{ Spectral sequence of dwarfs in training set M0-M9; chip 1 of APOGEE
spectrum with highlighted spectral type sensitive regions identified in \citealt{Desphande:2013}.}
\end{figure}



\subsection{Temperature/Metallicity Model}
\begin{figure}[ht]
\plottwo{figures/mann_teff_self_test.png}{figures/mann_feh_self_test.png}
\caption{Temperature and Metallicity label self test.}
\end{figure}

\begin{figure}[ht]
\plotone{figures/mann_model_demo.png}
\caption{\textit{Top two plots:} Mann-trained model for varying temperatures; \textit{bottom two plots:} Mann-trained model for varying metallicities.}
\end{figure}


%%==========================================================================================
\section{Discussion} \label{sec:discussion}

\begin{enumerate}
\item[-] Brief summary of what we did and found

\item[-] Things we learned along the way: For example, not all training sets worked well in our experiments, probably because they were too small or had too little dynamic range.

\item[-] Some discussion of precision and accuracy: The Cannon can produce precise results, but is only as accurate as its training set. 

\item[-] The Cannon as a test of evaluating training parameter accuracy? How to analyze this?... \color{red}{Maybe add training on theoretical models? Could be a test of seeing how well The Cannon could derive labels if there was a "ground truth" and tell us limitations of The Cannon model (i.e. we could say, for input spectra with zero label uncertainty, The Cannon can reproduce flux at each pixel with less than $XX\%$ error (comment on how well The Cannon reconstructs important features), and can reproduce labels to $XX\%$ uncertainty). Compare this to Mann and other traiing sets. Not hard to add, but having some issues with continuum normalization.}\color{black}

\item[-] Some discussion of whether the derivatives are sensible or surprising. \color{red}{Add derivative plots.}\color{black}

\item[-] Analysis of specific features? \color{red}{Can easily plot some line lists, but not sure where to go with the analysis... What features are important for M dwarfs? Compare derivative plots of specific lines between theory-trained models and data-trained models?}\color{black}

\item[-] Other assumptions / issues?
\end{enumerate}

%%==========================================================================================

\bibliographystyle{aasjournal}
\bibliography{ref}

\end{document}








\documentclass[onecolumn]{aastex62}

\usepackage{color}
\usepackage{amsmath}
\usepackage{enumitem}


\definecolor{bcolor}{RGB}{0, 51, 153}
\hypersetup{linkcolor=bcolor,citecolor=green,filecolor=cyan,urlcolor=bcolor}

\newcommand{\vdag}{(v)^\dagger}
\newcommand\aastex{AAS\TeX}
\newcommand\latex{La\TeX}

\received{\color{red} MM/DD/YY}
\revised{\color{red} MM/DD/YY}
\accepted{\color{red} MM/DD/YY}
\submitjournal{\color{red} JOURNAL}

\shorttitle{Birky \& Hogg}
\shortauthors{Birky \& Hogg}


\begin{document}

\title{DATA DRIVEN MODELS FOR APOGEE M DWARFS}


\correspondingauthor{Jessica Birky}
\email{jbirky@ucsd.edu}

\author[0000-0002-7961-6881]{Jessica Birky}
\affil{Center for Astrophysics and Space Science, University of California San Diego, La Jolla, CA 92093, USA}
\affil{Max-Planck-Institut f\"ur Astronomie, K\"onigstuhl 17, D-69117 Heidelberg}

\author[0000-0003-2866-9403]{David W. Hogg}
\affil{Max-Planck-Institut f\"ur Astronomie, K\"onigstuhl 17, D-69117 Heidelberg}
\affiliation{Center for Computational Astrophysics, Flatiron Institute, 162 Fifth Ave, New York, NY 10010, USA}
\affiliation{Center for Cosmology and Particle Physics, Department of Physics, New York University, 726
Broadway, New York, NY 10003, USA}
\affil{Center for Data Science, New York University, 60 Fifth Ave, New York, NY 10011, USA}


\begin{abstract}

The Cannon (Ness et al. 2015) is a flexible, data-driven spectral modeling and parameter inference framework, demonstrated on high-resolution Apache Point Galactic Evolution Experiment (APOGEE; $\lambda/\Delta\lambda\sim22,500$, $1.5-1.7 \mu m$) spectra of giant stars to estimate stellar labels (Teff, logg, [Fe/H], and chemical abundances) to precisions higher than the model-grid pipeline. The lack of reliable stellar parameters reported by the APOGEE pipeline for temperatures less than $\sim3550K$ (Schmidt et al. 2016), motivates the extension of this approach to M dwarf stars. Using a training set of 51 M dwarfs with spectral types ranging M0-M9 obtained from SDSS optical spectra, we demonstrate that The Cannon can infer spectral types to a precision of 0.6 types. We then use 30 M dwarfs ranging 3072 $<$ Teff $<$ 4131K, and -0.48 $<$ [Fe/H] $<$ 0.49 to train a two-parameter model precise to 44K and 0.05 dex respectively. Additionally we discuss the extension of a model to other labels, and the scientific objectives a data-driven pipeline could enable.

\end{abstract}

\keywords{stars: Mdwarfs -- stellar parameters}
 
%%==========================================================================================
\section{Introduction} \label{sec:intro}

\begin{enumerate}
\item[-] One paragraph about why M-dwarfs are so important in astronomy right now.

\item[-] Various attempts / samples for which people have tried to make substantial catalogs of M-dwarfs with physical labels on them. We might need help here!

\item[-] The idea in general of The Cannon, and the fact that it could be used to amplify a small number of labeled M-dwarfs in a training set into a huge number of labeled M-dwarfs, provided that there are overlapping objects. Make the point that The Cannon does not require us to have good M-dwarf models in the APOGEE wavelength range at all!
\end{enumerate}

%%==========================================================================================
\section{Data} \label{sec:data}

\begin{enumerate}
\item[-] Describe the APOGEE data. Note that it concentrates on giant stars but gets many M-dwarfs nonetheless.

\item[-] Describe the spectral-type training data. One paragraph plus citations. Include any cuts we apply.

\item[-] Describe the Teff, Fe/H training data. Same; Include any cuts.
\end{enumerate}

%%==========================================================================================
\section{The Cannon} \label{sec:cannon}
A few-paragraph summary of what The Cannon does, with references to Ness, Casey, Ho.


%%==========================================================================================
\section{Experimental Results} \label{sec:results}
\begin{enumerate}
\item[-] The spectral-type Cannon. Show the results on your AAS poster, and also show plots of derivatives wrt type, with maybe blow-ups on interesting features.

\item[-] The parameter Cannon. Same.
\end{enumerate}

%%==========================================================================================
\section{Discussion} \label{sec:discussion}

\begin{enumerate}
\item[-] Brief summary of what we did and found

\item[-] Things we learned along the way: For example, not all training sets worked well in our experiments, probably because they were too small or had too little dynamic range.

\item[-] Some discussion of precision and accuracy: The Cannon can produce precise results, but is only as accurate as its training set.

\item[-] Some discussion of whether the derivatives are sensible or surprising.

\item[-] Other assumptions / issues?
\end{enumerate}

%%==========================================================================================
\end{document}







